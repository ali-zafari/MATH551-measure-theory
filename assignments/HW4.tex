\documentclass[12pt, letterpaper]{article}
\usepackage[utf8]{inputenc}
\usepackage{amsmath, amssymb, amsthm}
\usepackage{mathtools}

\usepackage{fancyhdr}
\usepackage{lipsum} % For placeholder text, you can remove this in your actual document.

\usepackage[headings]{fullpage} % Set margins and place page numbers at bottom center
\usepackage[shortlabels]{enumitem} % Use a. in the enumerate
\usepackage{amsmath} % aligned equations
\usepackage{graphicx} % include figure
\usepackage{float} % usage of H for figure float
\usepackage{amssymb} % \blacksqure
\usepackage{xhfill} % fill horizontal line
\usepackage{sectsty} % section coloring
\usepackage{hyperref}
\usepackage{setspace}
\usepackage{bm}

\onehalfspacing
\subsectionfont{\color{blue}}  % sets colour of sections

\pagestyle{fancy}
\fancyhf{} % Clear header and footer fields
\renewcommand{\headrulewidth}{0.4pt} % Horizontal line under the header
\setlength{\headheight}{18pt}

\fancyhead[L]{\large MATH 551-001 \scriptsize Fall 2023}
\fancyhead[C]{}
\fancyhead[R]{\large \textbf{Ali Zafari - 800350381}} % Page number on the right side
\fancyfoot[R]{\thepage}

\newcommand{\Z}{\mathbb{Z}}
\newcommand{\R}{\mathbb{R}}
\newcommand{\C}{\mathbb{C}}
\newcommand{\F}{\mathbb{F}}
\renewcommand{\S}{\mathcal{S}}
\newcommand{\bigO}{\mathcal{O}}
\newcommand{\Real}{\mathcal{Re}}
\newcommand{\poly}{\mathcal{P}}
\newcommand{\mat}{\mathcal{M}}
\renewcommand{\L}{L}
\newcommand{\U}{U}
\DeclareMathOperator{\Span}{span}
\newcommand{\Hom}{\mathcal{L}}
\DeclareMathOperator{\Null}{null}
\DeclareMathOperator{\Range}{range}
\newcommand{\defeq}{\vcentcolon=}
\newcommand{\restr}[1]{|_{#1}}
\renewcommand{\inf}{\mathop{\mathrm{inf}\vphantom{\mathrm{sup}}}}

\newcommand{\measure}[1]{\textcolor{violet}{\fbox{\small#1 of Measure, Integration \& Real Analysis}}}
\newcommand{\suppmeasure}[1]{\textcolor{violet}{\fbox{\small#1 of Supplement for Measure, Integration \& Real Analysis}}}

\begin{document}

\subsection*{2\hspace{1pt}B\hspace{20pt}Exercise 18 }
$\forall x\in\R$ derivative of $f$ exists as
\begin{align*}
    f'(x)=\lim_{\delta\rightarrow0}\frac{f(x+\delta)-f(x)}{\delta}
\end{align*}
There exist $k\in\mathbb{N}$ such that $\frac{1}{k}<\delta$ for each $\delta>0$, thus if $\delta\rightarrow0$ then $k\rightarrow\infty$.
\\Let sequence of $g_1, g_2, \dots$ defined as:
\begin{align*}
    g_k(x)=\frac{f(x+\frac{1}{k})-f(x)}{\frac{1}{k}}
\end{align*}
Since differentiability implies continuity, $f(x)$ is a continuous function therefore $\frac{f(x+\frac{1}{k})}{\frac{1}{k}}$ and $\frac{f(x)}{\frac{1}{k}}$ are continuous functions and also measurable $\forall k\in\mathbb{N}$ by \measure{2.41}.\\\\Then $g_k$ is also measurable by \measure{2.46}.\\\\
As a result, since $f'(x)$ is a point-wise limit of $g_k$, we have $\forall x\in\R$
\begin{align*}
    f'(x)=\lim_{k\rightarrow\infty}g_k(x)
\end{align*}
then by \measure{2.48} $f'(x)$ is also Borel measurable.
\clearpage

\subsection*{2\hspace{1pt}B\hspace{20pt}Exercise 25}
Let $f_k\quad\forall k\in\mathbb{N}$ be defined as
\begin{align*}
    f_k(x):=f(x)+\frac{x}{k}
\end{align*}
$f_k(x)$ is trivially an strictly increasing function where its point-wise limit equals $f(x)$, i.e. $f(x)=\lim_{k\rightarrow\infty}f_k(x)$, and satisfies the conditions mentioned in the problem.

\clearpage

\subsection*{2\hspace{1pt}C\hspace{20pt}Exercise 2}
Let $w_k\quad\forall k\in E$ be defined as
\begin{align*}
    w_k:=\mu(\{k\})
\end{align*}
which satisfies the conditions mentioned in the problem, when $E_k:=\{k\}$
\begin{align*}
    \mu(E)=\mu(\bigcup_{k\in E} E_k)=\sum_{k\in E}\mu(E_k)=\sum_{k\in E}\mu(\{k\})=\sum_{k\in E}w_k
\end{align*}
\clearpage

\subsection*{2\hspace{1pt}C\hspace{20pt}Exercise 9}
\begin{itemize}
    \item $\mu+\nu:\S\rightarrow[0,\infty]$
    \item $(\mu+\nu)(\varnothing)=\mu(\varnothing)+\nu(\varnothing)=0$
    \item For every disjoint sequence of sets $E_1, E_2, \dots\in\S$:
    \begin{align*}
        (\mu+\nu)(\bigcup_k E_k)=&\mu(\bigcup_k E_k)+\nu(\bigcup_k E_k)\\
        =&\sum_k\mu(E_k)+\sum_k\nu(E_k)\\
        =&\sum_k \mu(E_k)+\sum_k\nu(E_k)\\
        =&\sum_k (\mu+\nu)(E_k)
    \end{align*}
\end{itemize}

\clearpage

\end{document}
