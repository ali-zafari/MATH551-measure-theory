\documentclass[12pt, letterpaper]{article}
\usepackage[utf8]{inputenc}
\usepackage{amsmath, amssymb, amsthm}
\usepackage{mathtools}

\usepackage{fancyhdr}
\usepackage{lipsum} % For placeholder text, you can remove this in your actual document.

\usepackage[headings]{fullpage} % Set margins and place page numbers at bottom center
\usepackage[shortlabels]{enumitem} % Use a. in the enumerate
\usepackage{amsmath} % aligned equations
\usepackage{graphicx} % include figure
\usepackage{float} % usage of H for figure float
\usepackage{amssymb} % \blacksqure
\usepackage{xhfill} % fill horizontal line
\usepackage{sectsty} % section coloring
\usepackage{hyperref}
\usepackage{setspace}
\usepackage{bm}

\onehalfspacing
\subsectionfont{\color{blue}}  % sets colour of sections

\pagestyle{fancy}
\fancyhf{} % Clear header and footer fields
\renewcommand{\headrulewidth}{0.4pt} % Horizontal line under the header
\setlength{\headheight}{18pt}

\fancyhead[L]{\large MATH 551-001 \scriptsize Fall 2023}
\fancyhead[C]{}
\fancyhead[R]{\large \textbf{Ali Zafari - 800350381}} % Page number on the right side
\fancyfoot[R]{\thepage}

\newcommand{\Z}{\mathbb{Z}}
\newcommand{\R}{\mathbb{R}}
\newcommand{\C}{\mathbb{C}}
\newcommand{\F}{\mathbb{F}}
\renewcommand{\S}{\mathcal{S}}
\newcommand{\bigO}{\mathcal{O}}
\newcommand{\Real}{\mathcal{Re}}
\newcommand{\poly}{\mathcal{P}}
\newcommand{\mat}{\mathcal{M}}
\renewcommand{\L}{L}
\newcommand{\U}{U}
\DeclareMathOperator{\Span}{span}
\newcommand{\Hom}{\mathcal{L}}
\DeclareMathOperator{\Null}{null}
\DeclareMathOperator{\Range}{range}
\newcommand{\defeq}{\vcentcolon=}
\newcommand{\restr}[1]{|_{#1}}
\renewcommand{\inf}{\mathop{\mathrm{inf}\vphantom{\mathrm{sup}}}}

\newcommand{\measure}[1]{\textcolor{violet}{\fbox{\small#1 of Measure, Integration \& Real Analysis}}}
\newcommand{\suppmeasure}[1]{\textcolor{violet}{\fbox{\small#1 of Supplement for Measure, Integration \& Real Analysis}}}

\begin{document}

\subsection*{3\hspace{1pt}A\hspace{20pt}Exercise 2}
First we show that $\int gd\delta_c=g(c)$ for any simple function $g=\sum_{i=1}^n e_i\chi_{E_i}$:
\begin{align*}
\int gd\delta_c=\int \left(\sum_{i=1}^n e_i\chi_{E_i}\right)d\delta_c=\sum_{i=1}^n e_i\delta_c({E_i})=\sum_{i=1}^n e_i\chi_c({E_i})(c)=g(c)
\end{align*}
where we used \measure{3.7}. 

% Then we need to show that every non-negative $\S$-measurable function can be written as a simple function.

Using approximation of an $\S$-measurable function by simple functions and Monotone Convergence Theorem (\measure{2.89, 3.11}), suppose $f_1, f_2, \dots$ is an increasing sequence of simple functions where $f=\lim_{k\rightarrow\infty}f_k$. Since each $f_k$ is a simple function, we have $\int f_kd \delta_c=f_k(c)$. 

And since $\int fd\delta_c=\lim_{k\rightarrow\infty}\int f_kd\delta_c=\lim_{k\rightarrow\infty}f_k(c)$, then we have $\int fd \delta_c=f(c)$.
\clearpage

\subsection*{3\hspace{1pt}A\hspace{20pt}Exercise 3}
We have $\{x\in X:f(x)>0\}=\lim_{k\rightarrow\infty}\{x\in X:f(x)>\frac{1}{k}\}$.
\begin{itemize}[label={}]
    \item \fbox{$\Rightarrow$}\\
    If $\int fd\mu>0$ then 
    \begin{align*}
        \int_{\{x\in X:f(x)>0} fd\mu>0
    \end{align*}
    therefore we must have $\mu(\{x\in X:f(x)>\})>0$ since otherwise will be a contradiction (integration over empty set being greater than zero). 

    \item \fbox{$\Leftarrow$}\\
    If $\mu(\{x\in X:f(x)>0\})>0$ then there exists $k_0\in\Z^+$ such that $\mu(\{x\in X:f(x)>\frac{1}{k_0}\})>0$. Therefore:
    \begin{align*}
        \int_X f d\mu\geq\int_{\{x\in X:f(x)>\frac{1}{k_0}\}} f d\mu\geq \frac{1}{k_0}\mu(\{x\in X:f(x)>\frac{1}{k_0}\})>0
    \end{align*}
\end{itemize}
\clearpage

\subsection*{3\hspace{1pt}A\hspace{20pt}Exercise 9}
Three properties of a measure should be verified for the proposed measure of $v(A)=\int \chi_Afd\mu=\int_Afd\mu$:
\begin{itemize}
    \item Since $v(A)=\int_Afd\mu$ is the supremum over lower Lebesgue sums, and each of the sums are non-negative, $v(A)$ must be non-negative as well. Therefore $v(A)\geq 0\quad\forall A\in\S$.
    
    \item Integral of $f$ over an empty set is zero, since the measure of that set is zero. Therefore $v(\varnothing)=0$.
    
    \item
    Suppose disjoint sequence of sets $E_1, E_1, \dots\in \S$:
    \begin{align*}
        v(\bigcup_i E_i)&=\int\chi_{\cup_i E_i}f d\mu\\
        &=\sum_i\int\chi_{E_i}f d\mu\\
        &=\sum_i v(E_i)
    \end{align*}

    
\end{itemize}
\clearpage

\subsection*{3\hspace{1pt}A\hspace{20pt}Exercise 15}
\begin{enumerate}[(a)]
    \item First we prove it for a simple function $g$. $g(x)=\sum_{i=1}^n c_i\chi_{E_i}(x)$ and $g_t(x)=\sum_{i=1}^n c_i\chi_{E_i}(x-t)=\sum_{i=1}^n c_i\chi_{t+E_i}(x)$. 

    Then computing the integral:
    \begin{align*}
        \int g_td\lambda&=
        \int\sum_{i=1}^n c_i\chi_{t+E_i}d\lambda\\
        &=\sum_{i=1}^nc_i\lambda(t+E_i)\quad \measure{3.15}\\
        &=\sum_{i=1}^nc_i\lambda(E_i) \quad \measure{2.7}\\
        &=\int\sum_{i=1}^n c_i\chi_{E_i}d\lambda\\
        &=\int gd\lambda
    \end{align*}

    Using approximation of an $\S$-measurable function by simple functions and Monotone Convergence Theorem (\measure{2.89, 3.11}), suppose $f^1, f^2, \dots$ is an increasing sequence of simple functions where $f=\lim_{k\rightarrow\infty}f^k$. Since each $f_k$ is a simple function, we have $\int f^k_t d\lambda=\int f^k d\lambda$. 
    
    Taking limit of both sides and using Monotone Convergence Theorem we have $\int f_td\lambda=\int fd\lambda$.

    \item First we prove it for a simple function $g$. $g(x)=\sum_{i=1}^n c_i\chi_{E_i}(x)$ and $g_t(x)=\sum_{i=1}^n c_i\chi_{E_i}(tx)=\sum_{i=1}^n c_i\chi_{\frac{1}{t}E_i}(x)$.
    Then computing the integral:
    \begin{align*}
        \int g_td\lambda&=
        \int\sum_{i=1}^n c_i\chi_{\frac{1}{t}E_i}d\lambda\\
        &=\sum_{i=1}^nc_i\lambda(\frac{1}{t}E_i)\quad \measure{3.15}\\
        &=\frac{1}{|t|}\sum_{i=1}^nc_i\lambda(E_i) \quad \measure{Exercise 2A.2}\\
        &=\frac{1}{|t|}\int\sum_{i=1}^n c_i\chi_{E_i}d\lambda\\
        &=\frac{1}{|t|}\int gd\lambda
    \end{align*}
    Using approximation of an $\S$-measurable function by simple functions and Monotone Convergence Theorem (\measure{2.89, 3.11}), suppose $f^1, f^2, \dots$ is an increasing sequence of simple functions where $f=\lim_{k\rightarrow\infty}f^k$. Since each $f_k$ is a simple function, we have $\int f^k_t d\lambda=\int f^k d\lambda$. 
    
    Taking limit of both sides and using Monotone Convergence Theorem we have $\int f_td\lambda=\int fd\lambda$.
\end{enumerate}
\clearpage

\end{document}
