\documentclass[12pt, letterpaper]{article}
\usepackage[utf8]{inputenc}
\usepackage{amsmath, amssymb, amsthm}
\usepackage{mathtools}

\usepackage{fancyhdr}
\usepackage{lipsum} % For placeholder text, you can remove this in your actual document.

\usepackage[headings]{fullpage} % Set margins and place page numbers at bottom center
\usepackage[shortlabels]{enumitem} % Use a. in the enumerate
\usepackage{amsmath} % aligned equations
\usepackage{graphicx} % include figure
\usepackage{float} % usage of H for figure float
\usepackage{amssymb} % \blacksqure
\usepackage{xhfill} % fill horizontal line
\usepackage{sectsty} % section coloring
\usepackage{hyperref}
\usepackage{setspace}
\usepackage{bm}

\onehalfspacing
\subsectionfont{\color{blue}}  % sets colour of sections

\pagestyle{fancy}
\fancyhf{} % Clear header and footer fields
\renewcommand{\headrulewidth}{0.4pt} % Horizontal line under the header
\setlength{\headheight}{18pt}

\fancyhead[L]{\large MATH 551-001 \scriptsize Fall 2023}
\fancyhead[C]{}
\fancyhead[R]{\large \textbf{Ali Zafari - 800350381}} % Page number on the right side
\fancyfoot[R]{\thepage}

\newcommand{\Z}{\mathbb{Z}}
\newcommand{\R}{\mathbb{R}}
\newcommand{\C}{\mathbb{C}}
\newcommand{\F}{\mathbb{F}}
\renewcommand{\S}{\mathcal{S}}
\newcommand{\bigO}{\mathcal{O}}
\newcommand{\Real}{\mathcal{Re}}
\newcommand{\poly}{\mathcal{P}}
\newcommand{\mat}{\mathcal{M}}
\renewcommand{\L}{L}
\newcommand{\U}{U}
\DeclareMathOperator{\Span}{span}
\newcommand{\Hom}{\mathcal{L}}
\DeclareMathOperator{\Null}{null}
\DeclareMathOperator{\Range}{range}
\newcommand{\defeq}{\vcentcolon=}
\newcommand{\restr}[1]{|_{#1}}
\renewcommand{\inf}{\mathop{\mathrm{inf}\vphantom{\mathrm{sup}}}}

\newcommand{\measure}[1]{\textcolor{violet}{\fbox{\small#1 of Measure, Integration \& Real Analysis}}}
\newcommand{\suppmeasure}[1]{\textcolor{violet}{\fbox{\small#1 of Supplement for Measure, Integration \& Real Analysis}}}

\begin{document}

\subsection*{4\hspace{1pt}A\hspace{20pt}Exercise 1} Let $c, p>0$:
\begin{align*}
    \mu(\{x\in X: |h(x)|\geq c\})&=\int_{\{x\in X: |h(x)|\geq c\}} d\mu\\
    &=\frac{1}{c^p}\int_{\{x\in X: |h(x)|\geq c\}} c^p\;d\mu\\
    &\leq\frac{1}{c^p}\int_{\{x\in X: |h(x)|\geq c\}} |h|^p\;d\mu\\
    &\leq\frac{1}{c^p}\int |h|^p\;d\mu
\end{align*}
\clearpage

\subsection*{4\hspace{1pt}A\hspace{20pt}Exercise 2}
Let $c>0$:
\begin{align*}
    \mu\left(\left\{x\in X: \left|h(x)-\int h\;d\mu\right|\geq c\right\}\right)&=\int_{\left\{x\in X: \left|h(x)-\int h\;d\mu\right|\geq c\right\}} d\mu\\
    &=\frac{1}{c^2}\int_{\left\{x\in X: \left|h(x)-\int h\;d\mu\right|\geq c\right\}} c^2\;d\mu\\
    &\leq\frac{1}{c^2}\int_{\{x\in X: |h(x)|\geq c\}} \left|h(x)-\int h\;d\mu\right|^2\;d\mu\\
    &\leq\frac{1}{c^2}\int \left|h(x)-\int h\;d\mu\right|^2\;d\mu\\
    &=\frac{1}{c^2}\int \left[|h|^2-2h(x)\int h\;d\mu+\left(\int h\;d\mu\right)^2\right] \;d\mu\\
    &=\frac{1}{c^2}\left[\int h^2\;d\mu-2(\int h\;d\mu)\int h\;d\mu+\left(\int h\;d\mu\right)^2 \int d\mu\right]\\
    &=\frac{1}{c^2}\left[\int h^2\;d\mu-2\left(\int h\;d\mu\right)^2+\left(\int h\;d\mu\right)^2\left(1\right)\right]\\
    &=\frac{1}{c^2}\left[\int h^2\;d\mu-\left(\int h\;d\mu\right)^2\right]
\end{align*}
\clearpage

\subsection*{4\hspace{1pt}A\hspace{20pt}Exercise 4}
Assume arbitrary large $n>0$. Let $I_1=(0,1)$ and $I_2=(1-\frac{1}{n}, 2-\frac{1}{n})$, their union will be $I_1\cup I_2=(0,2-\frac{1}{n})$.\\
Let $v>0$ be the Vitalli constant, then:
\begin{align*}
    v*I_1&=(-0.5v+0.5,0.5v+0.5)\\
    v*I_2&=(-0.5v+1.5-\frac{1}{n},0.5v++1.5-\frac{1}{n})
\end{align*}
to have at least one of $v*I_1$ or $v*I_1$ be able to cover the union of $I_1$ and $I_2$, we must have:
\begin{align*}
    -0.5v+1.5-\frac{1}{n}<0\\
    v>3-\frac{2}{n}
\end{align*}
or
\begin{align*}
    0.5v+0.5>2-\frac{1}{n}\\
    v>3-\frac{2}{n}
\end{align*}
Since $n$ was chosen arbitrarily large, the only way to have Vitali Covering Lemma be correct, is to have its constant at least 3.
\clearpage

\subsection*{4\hspace{1pt}A\hspace{20pt}Exercise 8}
To find $h^*(b)=\sup_{t>0}\frac{1}{2t}\int_{b-t}^{b+t}|h|$, we separately find it over partitions on $b$:
\begin{itemize}
    \item $\mathbf{b<0}$\\
    If  $\color{blue}b+t\leq0$ then $h^*(b)=0$.\\
    Else if $\color{blue}b+t>0$, we can assume two partitions:
    \begin{itemize}[label={}]
        \item $\mathbf{b+t>1}$
        \begin{align*}
            \frac{1}{2t}\int_0^1x\;dx=\frac{1}{4t}<\frac{1}{4(1-b)}
        \end{align*}
        \item $\mathbf{1>b+t>0}$
        \begin{align*}
            \frac{1}{2t}\int_0^{b+t}x\;dx=\frac{(b+t)^2}{4t}<\frac{1}{4(1-b)}
        \end{align*}
    \end{itemize}
    hence in this region $h^*(b)=\frac{1}{4(1-b)}$.
    
    \item $\mathbf{b>1}$\\
    If  $\color{blue}b-t\geq1$ then $h^*(b)=0$.\\
    Else if $\color{blue}b-t<1$, we can assume two partitions:
    \begin{itemize}[label={}]
        \item $\mathbf{b-t<0}$
        \begin{align*}
            \frac{1}{2t}\int_0^1x\;dx=\frac{1}{4t}<\frac{1}{4b}
        \end{align*}
        \item $\mathbf{0<b-t<1}$
        \begin{align*}
            \frac{1}{2t}\int_{b-t}^{1}x\;dx=\frac{1-(b-t)^2}{4t}
        \end{align*}
        To find the supremum of the above, we look for $t>0$ which makes the derivative zero:
        \begin{align*}
            t=\sqrt{b^2-1}
        \end{align*}
        then we replace it into the expression above:
        \begin{align*}
            \frac{1-(b-t)^2}{4t}\leq\frac{b-\sqrt{b^2-1}}{2}
        \end{align*}
    \end{itemize}

    \item $\mathbf{0<b<1}$\\
    In this case, $\frac{1}{2t}\int_{b-t}^{b+t}x\;dx=b$.
\end{itemize}
To conclude, the Hardy-Littlewood maximal function is:
\begin{equation}
h^*(b)=
    \begin{cases}
        \frac{1}{4(1-b)} & \text{if } b \in (-\infty,0)\\
        b & \text{if } b \in [0,1]\\
        \frac{b-\sqrt{b^2-1}}{2} & \text{if } b \in (1,\infty)\\
    \end{cases}
\end{equation}
\clearpage

\end{document}
