\documentclass[12pt, letterpaper]{article}
\usepackage[utf8]{inputenc}
\usepackage{amsmath, amssymb, amsthm}
\usepackage{mathtools}

\usepackage{fancyhdr}
\usepackage{lipsum} % For placeholder text, you can remove this in your actual document.

\usepackage[headings]{fullpage} % Set margins and place page numbers at bottom center
\usepackage[shortlabels]{enumitem} % Use a. in the enumerate
\usepackage{amsmath} % aligned equations
\usepackage{graphicx} % include figure
\usepackage{float} % usage of H for figure float
\usepackage{amssymb} % \blacksqure
\usepackage{xhfill} % fill horizontal line
\usepackage{sectsty} % section coloring
\usepackage{hyperref}
\usepackage{setspace}
\usepackage{bm}

\onehalfspacing
\subsectionfont{\color{blue}}  % sets colour of sections

\pagestyle{fancy}
\fancyhf{} % Clear header and footer fields
\renewcommand{\headrulewidth}{0.4pt} % Horizontal line under the header
\setlength{\headheight}{18pt}

\fancyhead[L]{\large MATH 551-001 \scriptsize Fall 2023}
\fancyhead[C]{}
\fancyhead[R]{\large \textbf{Ali Zafari - 800350381}} % Page number on the right side
\fancyfoot[R]{\thepage}

\newcommand{\Z}{\mathbb{Z}}
\newcommand{\R}{\mathbb{R}}
\newcommand{\C}{\mathbb{C}}
\newcommand{\F}{\mathbb{F}}
\renewcommand{\S}{\mathcal{S}}
\newcommand{\bigO}{\mathcal{O}}
\newcommand{\Real}{\mathcal{Re}}
\newcommand{\poly}{\mathcal{P}}
\newcommand{\mat}{\mathcal{M}}
\renewcommand{\L}{L}
\newcommand{\U}{U}
\DeclareMathOperator{\Span}{span}
\newcommand{\Hom}{\mathcal{L}}
\DeclareMathOperator{\Null}{null}
\DeclareMathOperator{\Range}{range}
\newcommand{\defeq}{\vcentcolon=}
\newcommand{\restr}[1]{|_{#1}}
\renewcommand{\inf}{\mathop{\mathrm{inf}\vphantom{\mathrm{sup}}}}

\newcommand{\measure}[1]{\textcolor{violet}{\fbox{\small#1 of Measure, Integration \& Real Analysis}}}
\newcommand{\suppmeasure}[1]{\textcolor{violet}{\fbox{\small#1 of Supplement for Measure, Integration \& Real Analysis}}}

\begin{document}

\subsection*{4\hspace{1pt}B\hspace{20pt}Exercise 1}
(Notation: $f_I=\frac{1}{|I|}\int_If$)\\\\
Since $f\in\mathcal{L}^1(\R)$, by \measure{4.10} there exits $b\in\R$ such that
\begin{align*}
    \lim_{t\downarrow0}\frac{1}{2t}\int^{b+t}_{b-t}|f-f(b)|=0
\end{align*}
on the other hand we have:
% \begin{align*}
%     \frac{1}{2t}\int^{b+t}_{b-t}|f-f_{[b-t,b+t]}|&\leq\sup\{|f(x)-f_{[b-t,b+t]}|:|x-b|\leq t\}\\
%     &=\sup\{|f(x)-\frac{1}{2t}\int^{b+t}_{b-t}f|:|x-b|\leq t\}
% \end{align*}
\begin{align*}
    \frac{1}{2t}\int^{b+t}_{b-t}|f-f_{[b-t,b+t]}|&=\frac{1}{2t}\int^{b+t}_{b-t}|f-\frac{1}{2t}\int^{b+t}_{b-t}f|\\
    &\leq \frac{1}{2t}\int^{b+t}_{b-t}|f-\frac{1}{2t}\int^{b+t}_{b-t}f(b)|\\
    &=\frac{1}{2t}\int^{b+t}_{b-t}|f-f(b)|
\end{align*}
where taking $\lim_{t\downarrow0}$ of both sides completes the proof.
\clearpage

\subsection*{4\hspace{1pt}B\hspace{20pt}Exercise 2}
(Notation: $f_I=\frac{1}{|I|}\int_If$)\\\\
Since $f\in\mathcal{L}^1(\R)$, by \measure{4.10} there exits $b\in\R$ such that
\begin{align*}
    \lim_{t\downarrow0}\frac{1}{2t}\int^{b+t}_{b-t}|f-f(b)|=0
\end{align*}
Now let $I$ be an interval such that $b\in I$. Then:
\begin{align*}
    \left|(\frac{1}{|I|}\int_If)-f(b)\right|&=\left|(\frac{1}{|I|}\int_If)-\frac{1}{|I|}\int_If(b)\right|\\
    &=\left|\frac{1}{|I|}\int_I(f-f(b))\right|\\
    &\leq\frac{1}{|I|}\int_I\left|f-f(b)\right|\\
    &\leq\frac{1}{|I|}\int^{b+|I|}_{b-|I|}\left|f-f(b)\right|
\end{align*}
therefore:
\begin{align*}
    \lim_{t\downarrow0}\sup\{f_I-f(b): I \text{ is an interval of length $t$ containing $b$} \}=0
\end{align*}
which implies the desired result:
\begin{align*}
    \lim_{t\downarrow0}\sup\{\frac{1}{|I|}\int_I\left|f-f_I\right|: I \text{ is an interval of length $t$ containing $b$} \}=0
\end{align*}
\clearpage

\subsection*{4\hspace{1pt}B\hspace{20pt}Exercise 3}
(Notation: $f_I=\frac{1}{|I|}\int_If$)\\\\
Since $f^2\in\mathcal{L}^1(\R)$, we can assume that in a local neighborhood of $b\in\R$ we also have $f\in\mathcal{L}^1(\R)$. We have chosen $b$ by \measure{4.10} such that:
\begin{align*}
    \lim_{t\downarrow0}\frac{1}{2t}\int^{b+t}_{b-t}|f-f(b)|=0
\end{align*}
and also:
\begin{align*}
    \lim_{t\downarrow0}\frac{1}{2t}\int^{b+t}_{b-t}|f^2-f^2(b)|=0
\end{align*}
which implies that $\lim_{t\downarrow0}\frac{1}{2t}\int^{b+t}_{b-t}f=f(b)$ and $\lim_{t\downarrow0}\frac{1}{2t}\int^{b+t}_{b-t}f^2=f^2(b)$.

We have
\begin{align*}
    \frac{1}{2t}\int^{b+t}_{b-t}|f-f(b)|^2=\frac{1}{2t}\int^{b+t}_{b-t}f^2 - 2f(b)\frac{1}{2t}\int^{b+t}_{b-t}f + f^2(b)
\end{align*}
where taking $\lim_{t\downarrow0}$ of both sides completes the proof.
\clearpage

\subsection*{4\hspace{1pt}B\hspace{20pt}Exercise 6}
Since $h\in\mathcal{L}^1(\R)$, by \measure{4.19} we can define $g:\R\rightarrow\R$ such that $g(s)=\int^{s}_{-\infty}h$ and $g'(b)=f(b)$ for almost every $b\in\R$.\\\\
By the hypothesis, since $\int^{s}_{-\infty}h=0$ for all $s\in\R$, then $g(s)=0$ for all $s\in\R$.\\\\
As a result, $g'(b)=0=h(b)$ for almost every $b\in\R$.
\clearpage

\end{document}
