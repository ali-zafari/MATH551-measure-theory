\documentclass[12pt, letterpaper]{article}
\usepackage[utf8]{inputenc}
\usepackage{amsmath, amssymb, amsthm}
\usepackage{mathtools}

\usepackage{fancyhdr}
\usepackage{lipsum} % For placeholder text, you can remove this in your actual document.

\usepackage[headings]{fullpage} % Set margins and place page numbers at bottom center
\usepackage[shortlabels]{enumitem} % Use a. in the enumerate
\usepackage{amsmath} % aligned equations
\usepackage{graphicx} % include figure
\usepackage{float} % usage of H for figure float
\usepackage{amssymb} % \blacksqure
\usepackage{xhfill} % fill horizontal line
\usepackage{sectsty} % section coloring
\usepackage{hyperref}
\usepackage{setspace}
\usepackage{bm}

\onehalfspacing
\subsectionfont{\color{blue}}  % sets colour of sections

\pagestyle{fancy}
\fancyhf{} % Clear header and footer fields
\renewcommand{\headrulewidth}{0.4pt} % Horizontal line under the header
\setlength{\headheight}{18pt}

\fancyhead[L]{\large MATH 551-001 \scriptsize Fall 2023}
\fancyhead[C]{}
\fancyhead[R]{\large \textbf{Ali Zafari - 800350381}} % Page number on the right side
\fancyfoot[R]{\thepage}

\newcommand{\Z}{\mathbb{Z}}
\newcommand{\R}{\mathbb{R}}
\newcommand{\C}{\mathbb{C}}
\newcommand{\F}{\mathbb{F}}
\renewcommand{\S}{\mathcal{S}}
\newcommand{\bigO}{\mathcal{O}}
\newcommand{\Real}{\mathcal{Re}}
\newcommand{\poly}{\mathcal{P}}
\newcommand{\mat}{\mathcal{M}}
\renewcommand{\L}{L}
\newcommand{\U}{U}
\DeclareMathOperator{\Span}{span}
\newcommand{\Hom}{\mathcal{L}}
\DeclareMathOperator{\Null}{null}
\DeclareMathOperator{\Range}{range}
\newcommand{\defeq}{\vcentcolon=}
\newcommand{\restr}[1]{|_{#1}}
\renewcommand{\inf}{\mathop{\mathrm{inf}\vphantom{\mathrm{sup}}}}

\newcommand{\measure}[1]{\textcolor{violet}{\fbox{\small#1 of Measure, Integration \& Real Analysis}}}
\newcommand{\suppmeasure}[1]{\textcolor{violet}{\fbox{\small#1 of Supplement for Measure, Integration \& Real Analysis}}}

\begin{document}

\subsection*{3\hspace{1pt}A\hspace{20pt}Exercise 2}
Suppose $f_1=e^{-|x|}, f_2=e^{\frac{-|x|}{2}}, f_3=e^{\frac{-|x|}{3}}, \dots$. 

Then we have
\begin{align*}
    \lim_{k\rightarrow\infty}f_k(x)=\lim_{k\rightarrow\infty}e^{\frac{-|x|}{k}}=0
\end{align*}
Now we calculate the limit of the integral:
\begin{align*}
    \lim_{k\rightarrow\infty}\int f_k\;d\lambda
    =\lim_{k\rightarrow\infty}\int_{-\infty}^{\infty} e^{\frac{-|x|}{k}}\;dx
    =\lim_{k\rightarrow\infty}2\int_{0}^{\infty} e^{\frac{-x}{k}}\;dx
    =\infty
\end{align*}
\clearpage

\subsection*{3\hspace{1pt}A\hspace{20pt}Exercise 3}
WLOG suppose $b\geq a$:
\begin{align*}
    \left|g(a)-g(b)\right|&=\left|\int_{(-\infty, a)}f\;d\lambda-\int_{(-\infty, b)}f\;d\lambda\right|\\
    &=\left|\int_{(-\infty, a)}f\;d\lambda-\int_{(-\infty, a)}f\;d\lambda-\int_{(a, b)}f\;d\lambda\right|\\
    &=\left|\int_{(a, b)}f\;d\lambda\right|\\
    &\leq \lambda((a,b))\sup_{(a,b)}|f|
\end{align*}
where the last inequality holds by \measure{3.25}.
\\
Since $f\in\mathcal{L}^1(\lambda)$, then we can assume $\sup_{(a,b)}|f|=M<\infty$.
\\\\
Then to have $\left|g(a)-g(b)\right|<\epsilon$, we must have $(b-a)M<\epsilon$.
\\
Therefore, for fixed $\epsilon>0$ it suffices that we choose $\delta>\frac{\epsilon}{M}$ to have $\left|g(a)-g(b)\right|<\epsilon$ for all $a,b\in\R$ such that $|b-a|<\delta$. And by doing so, $g$ is uniformly continuous on $\R$.
\clearpage

\subsection*{3\hspace{1pt}A\hspace{20pt}Exercise 5}
Let $f_1=f\chi_{[-1,1]}, f_2=f\chi_{[-2,2]}, f_3=f\chi_{[-3,3]},\dots$ be sequence of Borel measurable functions. We can see that
\begin{align*}
    \lim_{k\rightarrow \infty} f_k(x)=f(x)
\end{align*}
for any $x\in X$.\\


Also $|f|$ is Borel measurable function from $X$ to $[0,\infty]$ satisfying two conditions:
\begin{itemize}
    \item $\int |f|\; d\lambda<\infty$
    \item $|f_k(x)|\leq |f(x)|$
\end{itemize}
for arbitrary $k\in\Z^+$ and any $x\in X$.

By using Dominated Convergence Theorem we have:
\begin{align*}
    \lim_{k\rightarrow \infty} \int f_k\; d\lambda=\int f\; d\lambda
\end{align*}
\clearpage

\subsection*{3\hspace{1pt}A\hspace{20pt}Exercise 12}
Let $f_1=\frac{(1-x)\cos(x^{-1})}{\sqrt{x}}, f_2=\frac{(1-x)^2\cos(x^{-2})}{\sqrt{x}}, f_3=\frac{(1-x)^3\cos(x^{-3})}{\sqrt{x}},\dots$ be sequence of measurable functions from $(0,1)$ to $[-\infty, \infty]$. We can see that
\begin{align*}
    \lim_{k\rightarrow \infty} f_k(x)=0
\end{align*}
for any $x\in (0,1)$.\\

Now let $g_j(x)=\frac{1}{\sqrt{x}}\chi_{(\frac{1}{j},1)}$, where $g_j$ is a non-negative increasing sequence of functions from $(0,1)$ to $[0, \infty]$.
We also have
\begin{align*}
    g(x)=\lim_{j\rightarrow\infty}g_j(x)
\end{align*}
by using Monotone Convergence Theorem we have 
\begin{align*}
    \lim_{j\rightarrow\infty}\int_{(0,1)} g_j\;d\mu=\int_{(0,1)} g\;d\mu
\end{align*}
Suppose $g(x)=\frac{1}{\sqrt{x}}$. Then $g$ is a Borel measurable function from $(0,1)$ to $[0,\infty]$ satisfying two conditions:
\begin{itemize}
    \item $\int_0^1 |g|\; d\mu=\int_{(0,1)} g d\mu=\lim_{j\rightarrow\infty}\int_{(0,1)} g_j\;d\mu=\lim_{j\rightarrow\infty}\int_0^1 g_j=\lim_{j\rightarrow\infty}\int_0^\frac{1}{j} x^{-1/2}dx= \lim_{j\rightarrow\infty}2-2\sqrt{\frac{1}{j}}=2 <\infty$
    \item $|f_k(x)|\leq |g(x)|$
\end{itemize}
for arbitrary $k\in\Z^+$ and any $x\in X$.

By using Dominated Convergence Theorem we have:
\begin{align*}
    \lim_{k\rightarrow \infty} \int_{(0,1)} f_k\; d\lambda=\int_{(0,1)} 0\; d\lambda=0
\end{align*}
\clearpage

\end{document}
