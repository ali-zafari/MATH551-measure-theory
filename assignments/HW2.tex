\documentclass[12pt, letterpaper]{article}
\usepackage[utf8]{inputenc}
\usepackage{amsmath, amssymb, amsthm}
\usepackage{mathtools}

\usepackage{fancyhdr}
\usepackage{lipsum} % For placeholder text, you can remove this in your actual document.

\usepackage[headings]{fullpage} % Set margins and place page numbers at bottom center
\usepackage[shortlabels]{enumitem} % Use a. in the enumerate
\usepackage{amsmath} % aligned equations
\usepackage{graphicx} % include figure
\usepackage{float} % usage of H for figure float
\usepackage{amssymb} % \blacksqure
\usepackage{xhfill} % fill horizontal line
\usepackage{sectsty} % section coloring
\usepackage{hyperref}
\subsectionfont{\color{blue}}  % sets colour of sections

\pagestyle{fancy}
\fancyhf{} % Clear header and footer fields
\renewcommand{\headrulewidth}{0.4pt} % Horizontal line under the header
\setlength{\headheight}{18pt}

\fancyhead[L]{\large MATH 551-001 \scriptsize Fall 2023}
\fancyhead[C]{}
\fancyhead[R]{\large \textbf{Ali Zafari - 800350381}} % Page number on the right side
\fancyfoot[R]{\thepage}

\newcommand{\Z}{\mathbb{Z}}
\newcommand{\R}{\mathbb{R}}
\newcommand{\C}{\mathbb{C}}
\newcommand{\F}{\mathbb{F}}
\newcommand{\bigO}{\mathcal{O}}
\newcommand{\Real}{\mathcal{Re}}
\newcommand{\poly}{\mathcal{P}}
\newcommand{\mat}{\mathcal{M}}
\renewcommand{\L}{L}
\newcommand{\U}{U}
\DeclareMathOperator{\Span}{span}
\newcommand{\Hom}{\mathcal{L}}
\DeclareMathOperator{\Null}{null}
\DeclareMathOperator{\Range}{range}
\newcommand{\defeq}{\vcentcolon=}
\newcommand{\restr}[1]{|_{#1}}
\renewcommand{\inf}{\mathop{\mathrm{inf}\vphantom{\mathrm{sup}}}}

\newcommand{\measure}[1]{\textcolor{violet}{\fbox{\small#1 of Measure, Integration \& Real Analysis}}}
\newcommand{\suppmeasure}[1]{\textcolor{violet}{\fbox{\small#1 of Supplement for Measure, Integration \& Real Analysis}}}

\begin{document}

\subsection*{2\hspace{1pt}A\hspace{20pt}Exercise 3 }

\begin{align*}
    B\subseteq B=B\cap(A\cup A^c)=(B\cap A)\cup(B\cap A^c)=(B\cap A)\cup(B\setminus A)
\end{align*}
by using order-preserving property of outer measure:
\begin{align*}
    |B|\leq|(B\cap A)\cup(B\setminus A)|  
\end{align*}
and using subadditiviy property of outer measure:
\begin{align}
    |B|\leq|B\cap A|+|B\setminus A| 
    \label{eq:eq1}
\end{align}
we also have $A\cap B\subseteq A$, so again by subadditiviy property of outer measure we have $|A\cap B|\leq |A|$, hence by replacing in Eq. \ref{eq:eq1}:
\begin{align*}
    |B|\leq|A|+|B\setminus A| \\
    |B\setminus A|\geq |B|-|A|
\end{align*}

\clearpage

\subsection*{2\hspace{1pt}A\hspace{20pt}Exercise 6}\label{sec:2aex6}
We know from \measure{2.14} that $|[a,b]|=b-a$.
\begin{itemize}
    \item $\mathbf{|(a, b)|=b-a}$
    \begin{itemize}
        \item To show $|(a, b)|\leq b-a$:
            \begin{align*}
                (a,b)\subseteq[a,b]
                \Longrightarrow|(a,b)|\leq |[a,b]|
                \Longrightarrow|(a,b)|\leq b-a
            \end{align*}
        \item To show $|(a, b)|\geq b-a$:\\
        We have $[a,b]=\{a\}\cup\{b\}\cup(a,b)$ and by using subaditivity property of outer measure:
        \begin{align*}
            \underbrace{|[a,b]|}_{b-a}\leq\underbrace{|\{a\}|}_0+\underbrace{|\{b\}|}_0+|(a,b)|\Longrightarrow |(a,b)|\geq b-a
        \end{align*}
        Therefore, $|(a, b)|=b-a$.
    \end{itemize}
    
    \item $\mathbf{|[a, b)|=b-a}$
    \begin{itemize}
        \item To show $|[a, b)|\leq b-a$:
            \begin{align*}
                [a,b)\subseteq[a,b]
                \Longrightarrow|[a,b)|\leq |[a,b]|
                \Longrightarrow|[a,b)|\leq b-a
            \end{align*}
        \item To show $|[a, b)|\geq b-a$:\\
        We have $[a,b]=\{b\}\cup[a,b)$ and by using subaditivity property of outer measure:
        \begin{align*}
            \underbrace{|[a,b]|}_{b-a}\leq\underbrace{|\{b\}|}_0+|[a,b)|\Longrightarrow |[a,b)|\geq b-a
        \end{align*}
        Therefore, $|[a, b)|=b-a$.
    \end{itemize}
    
    \item $\mathbf{|(a, b]|=b-a}$
    \begin{itemize}
        \item To show $|(a, b]|\leq b-a$:
            \begin{align*}
                (a,b]\subseteq[a,b]
                \Longrightarrow|(a,b]|\leq |[a,b]|
                \Longrightarrow|(a,b]|\leq b-a
            \end{align*}
        \item To show $|(a, b]|\geq b-a$:\\
        We have $[a,b]=\{a\}\cup(a,b]$ and by using subaditivity property of outer measure:
        \begin{align*}
            \underbrace{|[a,b]|}_{b-a}\leq\underbrace{|\{a\}|}_0+|(a,b]|\Longrightarrow |(a,b]|\geq b-a
        \end{align*}
        Therefore, $|(a, b]|=b-a$.
    \end{itemize}
\end{itemize}
\clearpage

\subsection*{2\hspace{1pt}A\hspace{20pt}Exercise 10}
We know from \measure{2.14} that $|[0,1]|=1$.
\begin{itemize}
    \item To show $|[0,1]\setminus Q|\leq 1$:\\
    \begin{align*}
        [0,1]\setminus Q\subseteq[0,1]\Longrightarrow |[0,1]\setminus Q|\leq 1
    \end{align*}
    \item To show $|[0,1]\setminus Q|\geq 1$:\\
    By using the proved result of {\color{blue}2A Exercise 3} and knowing that $|Q|=0<\infty$:
    \begin{align*}
        |[0,1]\setminus Q|\geq|[0,1]|-\underbrace{|Q|}_{0}\Longrightarrow|[0,1]\setminus Q|\geq1
    \end{align*}
\end{itemize}
Therefore$|[0,1]\setminus Q|=1$.
\clearpage

\subsection*{2\hspace{1pt}A\hspace{20pt}Exercise 12}
    \begin{align*}
        F=\R\setminus \bigcup_{k=1}^\infty(r_k-\frac{1}{2^k},r_k+\frac{1}{2^k})
    \end{align*}
    \begin{enumerate}[label=(\alph*)]
        \item Let $A:=\bigcup_{k=1}^\infty\left(r_k-\frac{1}{2^k},r_k+\frac{1}{2^k}\right)$.\\
        We know from \suppmeasure{0.55} that union of open subsets in $\R$ is an open subset. Therefore, since $F$ is the complement of $A$ and must be a closet subset of $\R$ by \suppmeasure{0.61}.
        
        \item By definition, there exists no rational number in $F$.\\
        Let $a,b\in I\subseteq F$ where $a$ and $b$ are two arbitrary distinct irrational numbers ($a\neq b$). WLOG we assume $b>a$. Since $I$ is supposed to be an interval, then $(a,b)\subseteq I$.\\
        By \suppmeasure{0.30} there must exist a rational number in $(a,b)$, which is a contradiction with the definition of $F$.\\
        Therefore $a$ and $b$ cannot be distinct and $I$ contains at most one element.
        
        \item To be able to use the proved result of {\color{blue}2A Exercise 3} we must first show that $|A|<\infty$:
        \begin{align*}
            |A|=\left|\bigcup_{k=1}^\infty(r_k-\frac{1}{2^k},r_k+\frac{1}{2^k})\right|<\sum_{k=1}^\infty\left|(r_k-\frac{1}{2^k},r_k+\frac{1}{2^k})\right|=2<\infty
        \end{align*}
        So we have:
        \begin{align*}
            |F|=\left|\R\setminus \bigcup_{k=1}^\infty(r_k-\frac{1}{2^k},r_k+\frac{1}{2^k})\right|\geq|\R|-2 >\infty
        \end{align*}
    \end{enumerate}
\clearpage

\end{document}
